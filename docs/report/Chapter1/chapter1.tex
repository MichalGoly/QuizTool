\chapter{Background \& Objectives}

% This section should discuss your preparation for the project, including background
% reading, your analysis of the problem and the process or method you have followed
% to help structure your work.  It is likely that you will reuse part of your outline
% project specification, but at this point in the project you should have more to talk about.

\section{Background}
% What was your background preparation for the project? What similar systems did
% you assess? What was your motivation and interest in this project?

\subsection{Motivation}
I enjoy programming, and this project seemed to involve a lot of coding. I was also excited
to build a complex, real time system, while applying my previous software engineering
experience and learning new technologies at the same time. I knew I would be given
a lot of freedom to choose the most appropriate tools for the task, and incorporate
modern software development practices, including continuous integration and
containerised environments, to deliver good quality software. I was also keen on developing something that could be actually
useful to the university once I leave Aberystwyth. It is more motivating to develop
a product, while knowing it could be potentially used in real life, as opposed to being
forgotten after the submission. I have experienced being presented with wrong slides
during a lecture in my first year at university, so I was happy to address the problem
of a single session Qwizdom license with my tool.

\subsection{Technology Considerations}
\subsubsection{Programming Languages}
% - Initial proposal to create an Android based system for students -> React Native -> MEAN
The initial proposal was to create the classroom quiz system using Java\cite{3} to run it natively
on Android\cite{4} mobile devices. The lecturer was supposed the create quizzes on his teaching
machine, by interacting with the system using a web front end. Lecture slides were then supposed
to be broadcasted to his audience, and they could use their mobile phones to answers questions,
which would be then sent back to the lecturer for analysis. I have had previous experience with
native Android development, therefore this approach seemed like a reasonable option. The only problem was,
iOS\cite{5} devices are very popular in the United Kingdom, and developing an app for Android would exclude
a good percentage of students from being able to actively participate in lectures presented. I have
therefore started to think about alternative approaches.

The second possibility was to use React Native\cite{6}, a JavaScript\cite{7} framework allowing
developers to create mobile applications in JavaScript and compile it down to both iOS and Android.
This approach would still require the web front end for the lecturer to be developed, and a natural
choice would be to use React\cite{8} to keep the learning curve as low as possible.

The final alternative considered, was to develop the whole tool as a web application. This way both the
front end for lecturers and students could be developed using the same framework. Members of the audience
could participate in lectures by accessing the web application using web browsers installed
both on their mobile phones, regardless of the operating system, or even their laptops. I considered
both React and Angular 4\cite{9}, since I have already briefly used it before. React is a library
for developing user interface, whereas Angular is a web development framework. The only caveat with
using Angular is that the developer needs to learn TypeScript\cite{10}, which compiles down to JavaScript.

\subsubsection{WebSockets}
- Researched what is necessary to create a real time system - Sockets

\subsubsection{Prototyping}
- Socket.io chat tutorial
- Angular 4 tutorial
- MEAN socket.io tutorial

\subsection{Similar Tools}
- Qwizdom

\subsection{Internal vs External Hosting}
- local LSX container vs external cloud provider -> LDAP

\section{Analysis}
% Taking into account the problem and what you learned from the background work,
% what was your analysis of the problem? How did your analysis help to decompose
% the problem into the main tasks that you would undertake? Were there alternative
% approaches? Why did you choose one approach compared to the alternatives?
%
% There should be a clear statement of the objectives of the work, which you will
% evaluate at the end of the work.
%
% In most cases, the agreed objectives or requirements will be the result of a compromise
% between what would ideally have been produced and what was determined to be possible in
%  the time available. A discussion of the process of arriving at the final list is usually appropriate.
%
% As mentioned in the lectures, think about possible security issues for the project topic.
%  Whilst these might not be relevant for all projects, do consider if there are relevant
%   for your project. Where there are relevant security issues, discuss how they will
%   this affect the work that you are doing. Carry forward this discussion into relevant
%    areas for design, implementation and testing.
- MEAN stack \& Docker \& docker-compose
- Version Control
- build - Circle CI -> testing, autodeploy
- AWS Production environment (did not know of Elasticbeanstalk at this point)
- Google Single Sign On -> Security
- Identification of major problems
- Top level requirements of the system


\section{Process}
% You need to describe briefly the life cycle model or research method that you used. You
%  do not need to write about all of the different process models that you are aware of.
%  Focus on the process model that you have used. It is possible that you needed to adapt
%  an existing process model to suit your project; clearly identify what you used and how
%  you adapted it for your needs.
