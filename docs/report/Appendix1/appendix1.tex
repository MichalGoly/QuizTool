\chapter{Third-Party Code and Libraries}
%
% If you have made use of any third party code or software libraries, i.e. any code
% that you have not designed and written yourself, then you must include this appendix.
%
% As has been said in lectures, it is acceptable and likely that you will make use of
%  third-party code and software libraries. If third party code or libraries are used,
%   your work will build on that to produce notable new work. The key requirement is that we
%   understand what is your original work and what work is based on that of other people.
%
% Therefore, you need to clearly state what you have used and where the original material can be
% found. Also, if you have made any changes to the original versions, you must explain what you have changed.
%
% As an example, you might include a definition such as:
%
% Apache POI library - The project has been used to read and write Microsoft Excel files (XLS) as
%  part of the interaction with the client's existing system for processing data. Version 3.10-FINAL
%   was used. The library is open source and it is available from the Apache Software Foundation
% \cite{apache_poi}. The library is released using the Apache License
% \cite{apache_license}. This library was used without modification.

% server package.json
\subparagraph{body-parser v1.18.2}
MIT licensed Node.js middleware for parsing bodies of incoming HTTP requests. Used as part of the
\texttt{server\_node} container. [Online] Available: https://github.com/expressjs/body-parser, [Accessed: Apr. 20, 2018].

\subparagraph{express v4.15.2}
MIT licensed Node.js web development framework used to develop the back end of the Quiz Tool.
[Online] Available: https://github.com/expressjs/body-parser, [Accessed: Apr. 20, 2018].

\subparagraph{jsonschema v1.2.2}
MIT licensed JSON Schema validator is used to make sure JSON send together with certain requests to the
back end of the Quiz Tool is valid. [Online] Available: https://github.com/tdegrunt/jsonschema, [Accessed: Apr. 20, 2018].

\subparagraph{mongoose v5.0.7}
MIT licensed schema-based solution to model data in applications using MongoDB.
[Online] Available: http://mongoosejs.com/, [Accessed: Apr. 20, 2018].

\subparagraph{multer v1.3.0}
MIT licensed Node.js middleware for handling form-data used during file uploads in
the Quiz Tool. [Online] Available: https://github.com/expressjs/multer, [Accessed: Apr. 20, 2018].

\subparagraph{passport v0.4.0}
MIT licensed authentication Node.js middleware used during Quiz Tool authentication.
[Online] Available: http://www.passportjs.org/, [Accessed: Apr. 20, 2018].

\subparagraph{passport-google-oauth v1.0.0}
MIT licensed passport middleware Google authentication strategy used in Quiz Tool.
[Online] Available: https://github.com/jaredhanson/passport-google-oauth, [Accessed: Apr. 20, 2018].

\subparagraph{pdf-extract v1.0.11}
Node.js PDF manipulation library utilising command line utilities to extract text from PDF
presentations in the Quiz Tool. [Online] Available: https://github.com/nisaacson/pdf-extract, [Accessed: Apr. 20, 2018].

\subparagraph{pdf2img v0.5.0}
MIT licensed Node.js module for PDF to image conversion. Used to extract PNG images from slides uploaded
to the Quiz Tool. [Online] Available: https://github.com/fitraditya/node-pdf2img, [Accessed: Apr. 20, 2018].

\subparagraph{request-promise v4.2.2}
ISC licensed simplified HTTP request client with Promise support. Used during Google authentication in Quiz Tool.
[Online] Available: https://github.com/request/request-promise, [Accessed: Apr. 20, 2018].

\subparagraph{request v2.83.0}
Apache 2 licensed simplified Node.js HTTP request client used to make requests from the Quiz Tool.
[Online] Available: https://github.com/request/request, [Accessed: Apr. 20, 2018].

\subparagraph{socket.io v2.0.4}
MIT licensed socket.io engine used for bidirectional event-based communication between Quiz Tool's
back end, and clients. [Online] Available: https://github.com/socketio/socket.io/, [Accessed: Apr. 20, 2018].

\subparagraph{mocha v5.0.4}
MIT licensed test framefork for Node.js, it is used in back end unit testing of the Quiz Tool.
 [Online] Available: https://mochajs.org/, [Accessed: Apr. 20, 2018].

\subparagraph{chai v4.1.2}
MIT licensed assertion library used in the back end unit testing of the Quiz Tool.
 [Online] Available: http://www.chaijs.com/, [Accessed: Apr. 20, 2018].

\subparagraph{chai-http v3.0.0}
An integration testing library allowing creation of tests which can call endpoints
exposed by the Node.js applications.  [Online] Available: https://github.com/chaijs/chai-http, [Accessed: Apr. 20, 2018].

\subparagraph{selenium-webdriver v4.0.0-alpha.1}
The selenium-driver can be used to use Selenium to automate web browser automation. The
JavaScript implementation has been added to the tester container running integration tests of
the Quiz Tool. [Online] Available: https://github.com/SeleniumHQ/selenium/tree/master/javascript/node/selenium-webdriver, [Accessed: Apr. 20, 2018].

% server Dockerfile
\subparagraph{ghostscript}
One of the pdf-extract dependencies which is installed on the back end container, used for
extraction of text from PDF documents. [Online] Available:
http://git.ghostscript.com/ghostpdl.git/, [Accessed: Apr. 20, 2018].

\subparagraph{poppler-utils}
One of the pdf-extract dependencies which is installed on the back end container, for
getting information of PDF documents, convert them to other formats or manimulating them.
[Online] Available: https://packages.debian.org/sid/poppler-utils, [Accessed: Apr. 20, 2018].

\subparagraph{pdftk}
One of the pdf-extract dependencies which is installed on the back end container. PDFtk is
a command-line tool for working with PDFs. It is licensed under GPLv2.
[Online] Available: https://www.pdflabs.com/tools/pdftk-server/, [Accessed: Apr. 20, 2018].

\subparagraph{graphicsmagick}
MIT licensed node-pdf2img dependency, used for extraction of PNG images from PDF slides in the Quiz Tool.
[Online] Available: http://www.graphicsmagick.org/, [Accessed: Apr. 20, 2018].

% client package.json
\subparagraph{angular v4.3.5}
Angular is a TypeScript-based front end development framework supported by Google, used to
develop the front end of the Quiz Tool.
[Online] Available: https://angular.io/, [Accessed: Apr. 20, 2018].

\subparagraph{chart.js v2.7.2}
MIT licensed graph plotting library written in JavaScript, used for displaying the answers
students submit during quizzes.
[Online] Available: https://www.chartjs.org/, [Accessed: Apr. 20, 2018].

\subparagraph{file-saver v1.3.8}
MIT licensed library used for saving files on the client side.
[Online] Available: https://github.com/eligrey/FileSaver.js/, [Accessed: Apr. 20, 2018].

\subparagraph{core-js v2.5.0}
MIT licensed JavaScript standard library, automatically installed when a new Angular project
is installed. [Online] Available: https://github.com/zloirock/core-js, [Accessed: Apr. 20, 2018].

\subparagraph{font-awesome v4.7.0}
MIT licensed set of fonts used during the development of the Quiz Tool.
[Online] Available: https://github.com/FortAwesome/Font-Awesome, [Accessed: Apr. 20, 2018].

\subparagraph{intl v1.2.5}
MIT licensed compatibility implementation of the ECMAScript Internationalization API (ECMA-402) for JavaScript
installed as part of the Angular part of the tool. [Online] Available: https://github.com/andyearnshaw/Intl.js, [Accessed: Apr. 20, 2018].

\subparagraph{jspdf v1.3.5}
MIT licensed, client side JavaScript PDF generator used to generate PDF reports in the Quiz Tool.
[Online] Available: https://github.com/MrRio/jsPDF, [Accessed: Apr. 20, 2018].

\subparagraph{jspdf-autotable v2.3.2}
MIT licensed jdpdf plugin, used in Quiz Tool to generate tables in PDF reports.
[Online] Available: https://github.com/simonbengtsson/jsPDF-AutoTable, [Accessed: Apr. 20, 2018].

\subparagraph{mdi v2.1.19}
SIL OPEN FONT LICENSE Version 1.1 licensed distribution of the material desgin icons used in Quiz Tool.
[Online] Available: https://github.com/Templarian/MaterialDesign-Webfont, [Accessed: Apr. 20, 2018].

\subparagraph{ng2-charts v1.6.0}
MIT licensed chart.js wrapper for Angular, used to render audience responses as they come in during
broadcasts. [Online] Available: https://github.com/valor-software/ng2-charts, [Accessed: Apr. 20, 2018].

\subparagraph{ng2-file-upload v1.3.0}
MIT licensed Angular component used in Quiz Tool to handle file uploads on the client side.
[Online] Available: https://github.com/valor-software/ng2-charts, [Accessed: Apr. 20, 2018].

\subparagraph{ng2-materialize v1.8.0}
Apache v2 licensed Angular materializecss wrapper. Quiz Tool uses material components to achieve
a modern look and feel.
[Online] Available: https://github.com/sherweb/ng2-materialize, [Accessed: Apr. 20, 2018].

\subparagraph{ngx-cookie-service v1.0.10}
MIT licensed Angular service allowing easy access to cookies. Used in the Quiz Tool for
setting session cookies for authentication purposes.
[Online] Available: https://github.com/7leads/ngx-cookie-service, [Accessed: Apr. 20, 2018].

\subparagraph{rxjs v5.4.3}
Apache v2 licensed reactive programming library for JavaScript, used with the Angular HTTP
client to consume routes exposed by the back end of the tool.
[Online] Available: https://github.com/ReactiveX/rxjs, [Accessed: Apr. 20, 2018].

\subparagraph{socket.io-client v2.0.4}
MIT licensed socket.io client used to intercept slides in real time in Quiz Tool.
[Online] Available: https://github.com/socketio/socket.io-client, [Accessed: Apr. 20, 2018].

\subparagraph{zone.js v0.8.16}
MIT licensed Zones wrapper which comes with Angular applications.
[Online] Available: https://github.com/angular/zone.js, [Accessed: Apr. 20, 2018].

\subparagraph{codelyzer v3.1.2}
MIT licensed set of tslint rules for static code analysis of Angular TypeScript projects.
Used with the Atom text editor during the development of the Quiz Tool.
[Online] Available: https://github.com/mgechev/codelyzer, [Accessed: Apr. 20, 2018].

\subparagraph{jasmine-core v2.5.2}
MIT licensed JavaScript testing framework used during client side unit testing of the Quiz Tool.
[Online] Available: https://github.com/jasmine/jasmine, [Accessed: Apr. 20, 2018].

\subparagraph{jasmine-spec-reporter v3.2.0}
Apache v2 licensed real time console spec reporter for Jasmine testing framework, which comes
with Angular applications.
[Online] Available: https://github.com/bcaudan/jasmine-spec-reporter, [Accessed: Apr. 20, 2018].

\subparagraph{karma v1.4.1}
MIT licensed test runner for JavaScript used for Angular unit tests.
[Online] Available: https://github.com/karma-runner/karma, [Accessed: Apr. 20, 2018].

\subparagraph{karma-cli v1.0.1}
MIT licensed karma command line interface.
[Online] Available: https://github.com/karma-runner/karma-cli, [Accessed: Apr. 20, 2018].

\subparagraph{karma-coverage-istanbul-reporter v0.2.0}
MIT licensed karma test coverage reporter which has been installed together with karma but is
not actually used by the Quiz Tool.
[Online] Available: https://github.com/mattlewis92/karma-coverage-istanbul-reporter, [Accessed: Apr. 20, 2018].

\subparagraph{karma-jasmine v1.1.0}
MIT licensed karma plugin allowing it to work with the jasmine testing framework during front end unit tests of
the Quiz Tool.
[Online] Available: https://github.com/karma-runner/karma-jasmine, [Accessed: Apr. 20, 2018].

\subparagraph{karma-jasmine-html-reporter v0.2.2}
MIT licesed karma-jasmine reporter that dynamically shows tests results in a html format.
Used during local unit testing of the front end of the Quiz Tool.
[Online] Available: https://www.npmjs.com/package/karma-jasmine-html-reporter, [Accessed: Apr. 20, 2018].

\subparagraph{karma-phantomjs-launcher v1.0.4}
MIT licensed karma plugin which causes client side unit tests to be executed in a headless PhantomJS
web browser. [Online] Available: https://github.com/karma-runner/karma-phantomjs-launcher, [Accessed: Apr. 20, 2018].

\subparagraph{protractor v5.1.0}
MIT licensed end-to-end test framework which comes with Angular applications. Protractor is not
actually used for testing of the Quiz Tool, as end-to-end tests run using the selenium-webdriver directly.
[Online] Available: https://github.com/angular/protractor, [Accessed: Apr. 20, 2018].

\subparagraph{ts-node v3.3.0}
MIT licensed Node.js TypeScript executor, included as a development dependency of Angular projects.
[Online] Available: https://github.com/TypeStrong/ts-node, [Accessed: Apr. 20, 2018].

\subparagraph{tslint v5.6.0}
Apache v2 licensed static code analyser for the TypeScript programming language. Used during
the development of the Quiz Tool to check code's quality.
[Online] Available: https://github.com/palantir/tslint, [Accessed: Apr. 20, 2018].

\subparagraph{typescript v2.4.2}
Apache v2 licensed superset of JavaScript. The front end of the Quiz Tool is written in
TypeScript.
[Online] Available: https://github.com/Microsoft/TypeScript, [Accessed: Apr. 20, 2018].

% tutorials
\subparagraph{Building Chat Application using MEAN Stack (Angular 4) and Socket.io}
Step by step tutorial of building a simple chat application using MEAN stack and Socket.io. It helped me
to understand how to use MEAN with Socket.io together, and the initial structure of the application
was inspired by it.
[Online] Available: https://www.djamware.com/post/58e0d15280aca75cdc948e4e/building-chat-applicationusing-mean-stack-angular-4-and-socketio,
[Accessed: Apr. 11, 2018].

\subparagraph{Docker Compose | Containerizing MEAN Stack Application | DevOps Tutorial | Edureka}
A YouTube tutorial explaining how to containerise a MEAN stack web application. The \texttt{docker-compose.yml}
file has been based on the one showed in the video. [Online] Available: https://www.youtube.com/watch?v=WZa7GsqyS3w,
[Accessed: Apr. 20, 2018].

\subparagraph{Dockerized Angular 4 App (with Angular CLI)}
MIT licensed GitHub repository showing a starter Angular application, containerised with nginx
using Docker. The structure of \texttt{client} container is based on a fork of the repository.
[Online] Available: https://github.com/avatsaev/angular4-docker-example, [Accessed: Apr. 20, 2018].

\subparagraph{Multicontainer Docker Environments}
An official AWS tutorial showing how to deploy an application to the Multicontainer Elasticbeanstalk Docker
Environment. [Online] Available: https://docs.aws.amazon.com/elasticbeanstalk/latest/dg/create\_deploy\_docker\_ecs.html, [Accessed: Apr. 20, 2018].

\subparagraph{Test a Node RESTful API with Mocha and Chai}
A tutorial showing how to tests Node.js applications with Mocha and Chai. The structure of the server side
unit tests has been inspired by the tutorial. [Online] Available: https://scotch.io/tutorials/test-a-node-restful-api-with-mocha-and-chai,
[Accessed: Apr. 20, 2018].

\subparagraph{MEAN with Angular 2/5 - User Registration and Login Example \& Tutorial}
The authentication of the Quiz Tool has been inspired by this tutorial. Especially the Angular
JWT interceptor used in the application to append authentication tokens to each HTTP requests is based on the
code presented in the tutorial. [Online] Available: http://jasonwatmore.com/post/2017/02/22/mean-with-angular-2-user-registration-and-login-example-tutorial,
[Accessed: Apr. 20, 2018].

\subparagraph{6 16 integrating with google oauth with passport js in a MEAN app undergrad webdev summer 1 2017}
A YouTube tutorial explaining how to configure google authentication with the passport Node.js
authentication middleware, in a MEAN stack application. The Google Sign-In implemented in Quiz Tool
has been based on this tutorial. [Online] Available: https://www.youtube.com/watch?v=rc6zYV4jShQ,
[Accessed: Apr. 20, 2018].

\subparagraph{How to setup Elastic Beanstalk Deployment?}
A topic in the official Circle CI discussion forum describing how to setup Circle CI to automatically
deploy to the Elastic Beanstalk environment hosted on AWS. [Online] Available: https://discuss.circleci.com/t/how-to-setup-elastic-beanstalk-deployment/6154/4,
[Accessed: Apr. 20, 2018].

\subparagraph{Getting Started with Docker Compose}
A step-by-step introduction to using the official Selenium Docker images using docker-compose. The tutorial
was used when integration selenium tests were added to test the application.
[Online] Available: https://github.com/SeleniumHQ/docker-selenium/wiki/Getting-Started-with-Docker-Compose,
[Accessed: Apr. 20, 2018].

\subparagraph{Authorization type support}
A GitHub issue showing how to append an authorisation bearer token to the files uploader's requests.
[Online] Available: https://github.com/valor-software/ng2-file-upload/issues/317,
[Accessed: Apr. 20, 2018].











%
